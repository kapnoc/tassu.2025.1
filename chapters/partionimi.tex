\section{Partionimi?}

\textit{Partionimi on partiotoiminnassa käytetty lempinimi. Aiemmin perinne 
käyttää partionimiä oli huomattavasti yleisempää, mutta nykyään vain 
harvalla lippukuntalaisella on partionimi. Ehkä tämä juttu innostaa sinut 
miettimään, haluasitko tulla kutsutuksi partionimellä ja elvyttää hiipuvaa 
perinnettä?}

\begin{multicols}{2}
\noindent Joissakin lippukunnissa partiolainen päättää partionimensä itse, 
joissakin lippukunnissa taas muut partiolaiset antavat sen hänelle. Voi olla 
esimerkiksi, että uusi jäsen ''kastetaan'' hänen ensimmäisellä 
leirillään leirikasteessa hänelle valitulla partionimellä. Partionimi voi 
olla myös henkilön oikea kutsumanimi, jona hän on tunnettu myös partion 
ulkopuolella.

Joskus partionimet saattavat säilyä partiolaisella läpi elämän. Joidenkin 
partionimissä voi mennä pidempään vakiintua ja nimi saattaa muuttaa 
kirjoitusasuaan tai vaihtua kokonaan hänen partiopolkunsa aikana. Oleellista 
on, että partionimi tarvitsee käyttöä jäädäkseen kanssapartiolaisten 
huulille.

Itse partionimi voi olla melkein mitä vain, mutta varsin tyypllistä on, että 
se liittyy jotkenkin partiolaisen luonteenpiirteisiin, etu- tai sukunimeen. Sen 
taustalla voi olla myös jokin hauska lausahdus tai tapahtuma; partionimi voi 
syntyä hyvin yllättävässäkin tilanteessa.

Tärkein tehtävä partionimellä on yksilöidä kantajansa. Joskus lisänimen 
käyttö voi tuntua keinotekoiselta, jos partiolaiset ovat tekemisissä 
toistensa kanssa myös harrastuksen ulkopuolella. Samaan aikaan partionimillä 
pystytään välttämään etunimikaimojen tuomat haasteet.

{\smallskip\noindent\centering ***\par\smallskip}

Kurkisuon Rusakoissa oli hyvin pitkään perinteenä, että kukin uusi kolkka 
ja vartiolainen keksi itselleen partionimen. Tyypillisesti nimi oli jokin 
eläin kuten Pupu, Koala, Delfiini, Leopardi tai Kyyhky, mutta myös muita 
luontoon liittyviä nimiä kuten Karvakuono, Salama, Pottu, Kaarna ja Ruusu 
esiintyi. Joskus nimi tuli jostain muusta yleisnimestä kuten Pamppu, Keiju, 
Stoori, Napero ja Pilkku, kun taas joskus se juontui jostain aivan muualta 
kuten Jepulski, Nici, Eevi, Luru ja Inni. 

Nykyään lippukunnan jäsenistä löytyy ainakin Jääkarhu, Kala, Kike, 
Kirsku, Käärme, Leijona, Nonna, Ristilukki ja Rusakko. Tiedätkö sinä, 
keitä he ovat?
\end{multicols}

\vfill

\noindent\null\hfill Teksti: Ristilukki
