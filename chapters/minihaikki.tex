\section{Tarpojien minihaikki 14.-16.3.}

\begin{multicols}{2}

Perjantaina 14. maaliskuuta seitsemän rusakkoa lähti kohti Nuuksion kansallispuistoa viettämään mukavaa viikonloppua pienimuotoisen vaelluksen merkeissä. Vaikka vaellus oli nimensä mukaisesti tarpojille suunnattu, lopulta matkaan lähti vain kolme tarpojaa, Alden, Jetro ja Tesla, sekä neljä johtajaa, Ahti, Janne, Leo ja Tanguy. Retkikunta Leoa ja Jetroa lukuunottamatta lähti liikkeelle kello 17.00 Kontulasta, Mikaelinkirkolta. Kontulasta hypättiin metroon suuntana Rautatientori, josta matka jatkui E-junalla kohti Espoon keskusta. Leo ja Jetro hyppäsivät junan kyytiin Huopalahden asemalta ja näin koko retkikunta olikin kasassa! Junassa suoritettiin yhteisten varusteiden jakoa, nautittiin matkaevästä ja manailtiin seuraavan yön sääennustetta, joka lupaili jopa 9 asteen pakkasta. Onneksi kaikilla oli varusteet kunnossa, jolloin kylmyys ei pääsisi haittaamaan! Junamatkan päätteeksi saavuttiin Espoon keskukseen, josta bussi 245A lähti kuljettamaan retkeilijöitä kohti kansallispuistoa. Bussimatka kesti kaikkiaan noin 20 minuuttia, minkä jälkeen retkikunta jäi pois Kaitakorven bussipysäkillä. 

Ensimmäisen noin kahden kilometrin etapin suunnistamisen sai vastuulleen Jetro, jonka tehtävänä oli suunnistaa porukka Hynkänlammen leiripaikalle. Reitti Kaitakorvelta Hynkänlammelle kulki läpi Nuuksion ratsastuskeskuksen tilusten. Hevosaitauksen vieressä kulkevaa polkua kävellessä retkikunnasta alkoi tuntua, kun joku tuijottaisi heitä. Muutama ulkona oleileva hevonen ei selkeästi ollut varautunut tunkeilijoihin, vaan tuijotti retkeilijöitä ihmeissään. Hevoskohtaamisen jälkeen matka jatkui pienen nousun kautta ylös Hynkänlammen keittokatokselle, joka oli retkikunnan ensimmäinen yöpaikka. Oli jo ilta ja pimeää, minkä vuoksi retkikunta aloitti majoitteiden pystyttämisen miltei saman tien. Kun muut lähtivät kasaamaan laavuja, telttaa tai riippumattoa, jäi Leo tekemään tulia iltapalaa varten. Puut olivat kuitenkin märkiä eikä Leo ymmärtänyt kasata nuotiota ylemmälle ritilälle tulipaikan pohjan sijaan, jolloin tulen tekemisessä kesti odotettua kauemmin. Lopulta Ahti sai tulen roihuamaan ja iltapalaksi nautittiin teetä ja makkaraa, minkä jälkeen väsynyt retkikunta painui unten maille!

Aamulla herätessä todettiin, että sääennuste oli hyvinkin pitänyt paikkansa, sillä ainakin allekirjoittanut voi todeta, että kylmyys tosiaan kiusasi. Yöllä oltiin vältytty lumisateelta, mutta aamulla lämpötilan noustessa kosteus oli hiipinyt sisään telttaan, jolloin varusteet olivat kostuneet. Aamupalalla sitten kuivateltiin varusteita nuotion ääressä, puuroa, kahvia ja Oivallusta nautiskellen. Pian olikin jo aika purkaa majoitteet ja jatkaa matkaa. Ensimmäisen suunnistusvuoron sai Leo, joka pienen harhailun jälkeen sai suunnistettua porukan lounaspaikalle. Noin kahden kilometrin kävelyn jälkeen retkikunta pysähtyi Kattilajärventien varrelle valmistamaan lounasta. Trangioilla valmistui herkullinen “ranskalainen” sipulikeitto, jonka mausteena toimi kuivattu sipuli. Ja mitä olisikaan ateria ilman Oivallusta ja kahvia! Lautaset tyhjinä ja energiaa täyteen tankattuina retkeilijät lähtivät tarpomaan kohti seuraavaa etappia, Urja-järveä. Seuraavan suunnistusvuoron sai Alden, joka sunnistikin onnistuneesti muutaman kilometrin matkan Urjalle, jonka rannalla sijaitsikin jo seuraava yöpaikka. Matkalla päiviteltiin jälleen seuraavan yön sääennustetta, joka vaikutti aiempaakin yötä karmeammalta. Yksi aste lämmintä ja räntää… 

Urjalle saavuttaessa todettiin, ettei laavupaikan löytäminen ollut yhtään niin helppoa kuin Hynkänlammella ja sopivia paikkoja etsittiin miltei puoli tuntia. Kun paikat löytyivät ja majoitteet saatiin kasaan, oli jälleen ruoka-aika. Koska Urjalla oli rusakoiden lisäksi muitakin retkeilijöitä, jotka olivat miehittäneet nuotiopaikan, täytyi retkikunnan valmistaa ruokaa trangialla. Iltaruuaksi syötiin tomaattista soijapataa sekä salami-tomaattipastaa. Kasvisruokailijoiden harmiksi soijanpalaset maistuivat pitkälti pahvilta, mutta siitä huolimatta vatsat saatiin täyteen. Loppuilta kuluikin yhdessä aikaa viettäessä ja jutellessa. Iltapalana oli Hönöä, Oivallusta, Lämmintä kuppia ja teetä. Illan pimentyessä ja kylmetessä retkikunta painui nukkumaan.

Haikin viimeinen aamu alkoi kosteissa tunnelmissa. Sääennuste oli totta tosiaan pitänyt paikkansa ja yön räntäsade oli tullut läpi laavun katosta. Etenkin teltan keskiosaan kerätyt tavarat kuten allekirjoittaneen toppatakki olivat kastuneet. Onneksi muulta seurueelta sai kuivaa vaatetta lainaan, minkä lisäksi aamupuuro ja -kahvi sekä nuotio lämmittivät aamun kylmyydessä. Aamiaisen jälkeen oli vielä aika tiskata astiat, pakata loput tavarat ja purkaa majoitteet, minkä jälkeen matka pääsi jatkumaan. Suunnistusvuoron sai nyt puolestaan Tesla, jonka tehtävänä oli suunnistaa seurue Haltiaan bussipysäkille. Matkalla päädyttiin kuitenkin tekemään pieni poikkeus, kun retkikunta päätti lähteä katsomaan Meerlammen luolaa. Matkalla luolalle ylitettiin pieniä vesialueita, listattiin suomen kielen mö-alkuisia sanoja ja nautittiin auringonpaisteesta, jota oli kaivattu koko viikonlopun ajan.

Meerlammella todettiin, että “luola” on hiukan harhaanjohtava termi paikalla olevista kallioista ja suurista siirtolohkareista. Se oli kuitenkin mukava levähdyspaikka ja sen kivimuodostelman alla saatiin otettua myös retken virallinen yhteiskuva. Meerlammen nähtävyyksien jälkeen matka jatkui kohti Haltiaa. Matkalla aloitettiin jälleen uusi kielellinen haaste, kun retkikunta alkoi keksimään tulevan kesän vaellukselle v-kirjaimella alkavia nimiä. Juomaveden loppuessa osa alkoi maistella Nuuksion purojen virtavesien tarjontaa, mikä herätti muissa hiukan ihmetystä. Lopulta saavuttiin Haltiaan, jolloin todettiin koko minihaikin matkasaldoksi yksitoista kilometriä. 245A-bussia saatiin odotella jonkin aikaa, jolloin oli hyvin aikaa käydä vielä vessassa Haltian luontokeskuksessa. Bussimatka saatiin jakaa kahden muun Nuuksiossa retkeilleen lippukunnan kanssa tupaten täydessä bussissa. Espoon keskuksesta lähdettiin E-junalla kohti kotia. Ikävistä sääolosuhteista huolimatta retki oli onnistunut ja seurue jälleen valmiina uusiin seikkailuihin.

\end{multicols}

\vfill

\noindent\null\hfill Teksti: Leo
