\section{Kesäretkeilyvinkit}

\begin{multicols}{2}

\noindent Kesä alkaa olemaan nurkan takana ja siispä alkaa myös retkeilykausi! Kuitenkaan ei pidä lähteä soitellen sotaan, onhan kesähelteillä omat oikkunsa jotka tulee kaikkien ottaa huomioon. Sää voikin siis olla kuvankaunis ja mieli tekisi lähteä noin vain retkelle, on hyvä huomioida aurinkoisen sään kääntöpuolet, joista salakavalimmat ovat korkeat lämpötilat sekä voimakas UV-säteily. Tässäpä siis tärkeimpiä vinkkejä, joilla idyllisestä keskikesän hellepäivästä pääsee nauttimaan täysin rinnoin!

\subsection*{Auringolta suojautuminen}

\textbf{Käytä hattua}: lippalakki on hyvä, mutta niskan saa suojattua erinomaisesti lierihatulla!\\
\textbf{Aurinkovoide}: SPF 50 on hyvä, muista nenä ja korvantaukset.\\
\textbf{Aurinkolasit}: aina näppärät, mutta kaupasta saa myös silmälaseihin kiinnitettäviä aurinkolasilinssejä heille, jotka silmälaseja käyttävät. [toim. huom. Suosittelen!]\\
\textbf{Vaatetus}: pitkälahkeiset ja -hihaiset vaatteet suojaavat auringossa palamiselta, mutta muista pukea väriltään vaaleat vaatteet: tummat vaatteet sitovat lämpöä itseensä! Jos haluat pukea shortsit ja t-paidan, muista erityisesti tällöin laittaa kunnolla aurinkorasvaa.\\

{\centering\includegraphics[width=0.7\textwidth,trim={0 0 0 0},clip,angle=90]{assets/kesäretkeily1}}

\subsection*{Reitin valinta ja retkeilykohteet}

\textbf{Reitti}: jos retki on yhtään pidempi, ei helteellä ole hyväksi kävellä aurinkoisinta ja avointa tietä pitkin. Tällöin varjoisampi metsäpolku voi olla tervetullut\\
\textbf{Kohteet}: kesällä kannattaa harkita jonkin vesistön äärelle suuntaamista, sillä vesistöt viilentävät mukavasti ympäristöään niin, ettei tule tukalan kuuma. Myös mahdolliset kävelyretket voi suunnata kiertämään jonkin pienen järven tai seuraamaan rantaviivaa.\\
\textbf{Ajankohta}: keskikesällä retket on hyvä ajoittaa aamuun tai iltapäivään niin, ettei pahimpina keskipäivän tunteina joudu kärventymään auringossa. \\

\subsection*{Sekalaiset (mutta silti tosi tärkeät!) vinkit}

\textbf{Nesteytys}: ehkäpä kaikista tärkein muistettava asia kesäisin. Muista aina vara mukaan vettä, eikä mieluusti siinä kaikista pienimmässä pullossa! Kannattaa siis varata mukaan hieman enemmän juotavaa kuin uskoo tarvitsevansa, Auringon paahde voi yllättää.\\
\textbf{Yöpyminen}: jos retkellänsä haluaakin yöpyä, on hyvä ottaa huomioon muutama asia, kuten varjoisa leiripaikka, ohut makuupussi sekä hyttysverkko. Nämä pätevät majoitteeseen kuin majoitteeseen.

\vspace*{0.32cm}
{\centering\includegraphics[width=0.48\textwidth,trim={2cm 1cm 2cm 1cm},clip]{assets/kesäretkeily2}}

\subsection*{Kirjoittajan omat vinkit ja huomiot}

Vesistöissä on mukava vilvoitella, mutta myös märkä pyyhe/vaatekappale painettuna niskaa tai rannetta vasten on oiva keino viilentää kehoa: näillä alueilla verenkierto on suurta ja tätä tietoa onkin hyvä käyttää helteellä hyödykseen!

Hyttysmyrkky on aina metsään mentäessä hyvä, mutta hyttyshattu on kesäisin erinomainen kapistus: se suojaa kasvoja sekä niskaa Auringolta sen lisäksi, etteivät hyttyset pääse puremaan.

Jotain pientä purtavaa on aina hyvä olla mukana, mutta helteellä erityisen maistuvia ovat kaikenlaiset marjat ja miksei vaikka vesimeloni, vaikka tällaiset onkin ehkä helpompi ottaa mukaan piknikille kuin eräretkelle.

Retken päätteeksi, ja miksei sen aikanakin jos kyseessä on pidempi reissu, on tärkeää tutkia itsensä punkkien varalta, sillä ne jos mitkä ikäviä ja salakavalia ovat. Erityisen hyvä on muistaa tarkistaa polvitaipeet ja nivuset, sillä näihin paikkoihin punkkien on helppo piiloutua.

Retkelle sopii oikein hyvin lähteä kaverin kanssa, sillä onhan jaettu ilo kaksinkertainen ilo!

\vspace*{0.32cm}
\noindent\null\hfill Kuvat: Tanguy\\
\noindent\null\hfill Teksti: Elias

\end{multicols}

